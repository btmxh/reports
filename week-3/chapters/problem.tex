Trong quá trình phát triển những bộ môn của Toán học, rất nhiều những đối tượng đã được nghiên cứu rất nhiều từ xưa đến nay. Có những đối tượng có sự tương đồng lớn, do đó dẫn đến việc chúng có nhiều tính chất và ứng dụng giống nhau. 

Do vậy, trong toán học xuất hiện rất nhiều những định lý với cách chứng minh y hệt nhau, chỉ khác nhau về đối tượng đang được xem xét. Để giảm thiểu đi những công sức thừa này, một ngành toán học ra đời, đó là \textbf{đại số trừu tượng}.

Thay vì nghiên cứu đến những đối tượng cụ thể, ngành toán học này nghiên cứu các \textbf{cấu trúc đại số} như nhóm (group), vành (ring\footnote{Trong chương trình đại số của Trường Đai Học Bách Khoa Hà Nội, tương ứng với khái niệm vành trong tiếng Anh là các \textbf{rngs}, còn \textbf{rings} là những vành có đơn vị.}), trường (field).

Khi đó nhiều loại đối tượng sẽ cùng tương ứng với cùng một loại cấu trúc đại số, tạo ra mối quan hệ giữa nhiều mảng khác nhau của toán (nói riêng) và các ngành khoa học khác (nói chung).

Những cấu trúc nhóm và vành, do có những tính chất thú vị và nhiều ứng dụng thực tiễn, đã có riêng cho mình một ngành toán học, lần lượt là lý thuyết nhóm (group theory) và lý thuyết vành (ring theory). Lý thuyết nhóm được nghiên cứu rất kĩ và mang lại nhiều ứng dụng quan trọng trong vật lý (nghiên cứu mô hình hạt nhân nguyên tử, mô hình chuẩn của vật lý hạt nhân), khoa học máy tính (thuật toán giải Rubik, thuật toán đẳng cấu đồ thị). hóa học (những đối xứng trong hóa học nguyên tử).

