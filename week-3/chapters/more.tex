
Một bước phát triển tương đối mới (xuất hiện từ giữa thế kỷ XX) của các lý thuyết về các cấu trúc đại số là lý thuyết phạm trù (category theory). Tại đây, những phạm trù và những mối quan hệ giữa chúng được nghiên cứu. Ta có những phạm trù của tập hợp, của nhóm, và thậm chí phạm trù của các phạm trù.

Ngành toán học này trở thành góp phần tạo nên nền tảng của các hệ thống kiểu (type system) trong nhiều ngôn ngữ lập trình hiện đại, đặc biệt là những ngôn ngữ lập trình hàm (functional programming language) như Haskell. Một điều thú vị là một khái niệm của ngành này: "a monad is a monoid in the category of endofunctors" (một monad là một vị nhóm trong phạm trù của các endofunctor) cũng đã trở thành một Internet meme vì sự ngắn gọn mà khó hiểu của nó.

Một ngành toán học khác (mà không liên quan lắm đến các cấu trúc đại số) cũng trở thành một phần không thể thiếu được trong nhiều ngôn ngữ khác là \textbf{phép tính lambda} (lambda calculus), và có lẽ cái tên sẽ gợi ra sự quen thuộc đến nhiều lập trình viên.